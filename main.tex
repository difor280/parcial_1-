\documentclass{article}
\usepackage[utf8]{inputenc}
\usepackage[spanish]{babel}
\usepackage{listings}
\usepackage{graphicx}
\graphicspath{ {images/} }
\usepackage{cite}

\begin{document}

\begin{titlepage}
    \begin{center}
        \vspace*{1cm}
            
        \Huge
        \textbf{PARCIAL 1}
            
        \vspace{0.5cm}
        \LARGE
        INFORME
            
        \vspace{1.5cm}
            
        \textbf{Diego Fernando Urbano Palma\\Michael Stiven Zapata Giraldo\\Brayan Steven Avila Marin}
            
        \vfill
            
        \vspace{0.8cm}
            
        \Large
        Despartamento de Ingeniería Electrónica y Telecomunicaciones\\
        Universidad de Antioquia\\
        Medellín\\
        Abril de 2021
            
    \end{center}
\end{titlepage}

\tableofcontents
\newpage
\section{ Analisis del problema }\label{intro}
Para enfrentar el problema planteado en cual debemos crear una matriz de leds en los cuales debemos encenderlos y apagarlos para representar patrones ingresados por el usuario proponemos conectar los leds de manera multiplexada y hacer uso de transistores a manera de swicht para controlar filas y columnas de leds, esto controlado a su ves por medio de el circuito integrado 74HC595.
\section{ problemas a solucionar }
\begin{itemize}
\item Comprender el funcionamiento de los transistores para el control de filas y columnas de leds.
\item Conectar la matriz al circuito integrado 74HC595 de tal forma que podamos controlar el encendido de todos los leds.
\item Entender cómo se manipulan los leds por medio de la programación del Arduino.
\item Buscar las funciones más óptimas en c++ para hacer las amena la escritura y estructura del codigo



\end{itemize} 
\end{document}
